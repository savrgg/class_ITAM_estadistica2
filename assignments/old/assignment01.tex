%Example of use of oxmathproblems latex class for problem sheets
\documentclass{oxmathproblems}

%(un)comment this line to enable/disable output of any solutions in the file
%\printanswers

%define the page header/title info
\course{ITAM - Estadistica 2}
\oxfordterm{Assignment 01}
\sheetnumber{1}
\sheettitle{Repaso Estadística 1}

\begin{document}

En esta tarea se exploran básicos de estadística 1. Para los ejercicios \textbf{suponga muestras independientes.}

\begin{questions}

\miquestion \textbf{Propiedades de esperanza y varianza:}
Sean X, Y, Z variables aleatorias y usando las propiedades de la esperanza y varianza encuentre: (Use los siguientes valores: $E(X) = 8, V(X) = 10, E(Y) = 3, V(Y) = 5, E(Z) = -3, E(Z^2) = 10$)  
\begin{parts}
  \part $E(3X+5) = $
  \part $E(Z+Y) = $
  \part $V(X+Y+5) = $ 
  \part $V(X-Y-5) = $ 
  \part $V(Z) = $ 
  \part $E(Y^2) = $ 
  \part $E(3X^2+5) = $ 
  \part $E(X^2+Z^2+Y^2) = $
\end{parts}

\begin{solution}
  The solution would go here
\end{solution}


\miquestion \textbf{Cálculo de esperanza y varianza:}
Resuelva el ejercicio y determine cuando utilizar formulas para \textbf{datos agrupados} o \textbf{datos no agrupados}. 

\begin{parts}

\part X tiene la siguiente distribución poblacional:

\begin{center}
\begin{tabular}{ |c|c| } 
 \hline
 X=x & p(X = x)  \\ 
 1 & 0.2  \\ 
 2 & 0.35  \\ 
 3 & 0.2  \\ 
 4 & 0.25  \\ 
 \hline
\end{tabular}
\end{center}

Determine si son datos agrupados o no y encuentre E(X) y V(X)

\part Se tiene la siguiente muestra aleatoria: X = \{3, 5, 11, 5, 3\}. Determine si son datos agrupados o no y encuentre la $E(X)$ y $V(X)$.

\end{parts}

\begin{solution}
  The solution would go here
\end{solution}


\end{questions}

\end{document}
