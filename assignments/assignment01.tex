%Example of use of oxmathproblems latex class for problem sheets
\documentclass{oxmathproblems}
\usepackage{blindtext}
\usepackage{hyperref}
%(un)comment this line to enable/disable output of any solutions in the file
%\printanswers

%define the page header/title info
\course{ITAM - Estadistica 2}
\oxfordterm{Assignment 01}
\sheetnumber{1}
\sheettitle{}

\begin{document}

\begin{questions}

\miquestion \textbf{Datos no agrupados} Se presentan 5 datos de una muestra de precios de vuelos redondos a una playa mexicana en un viernes específico: 2500, 2000, 1800, 3000, 2100

\begin{itemize}
\item ¿Qué fórmulas ocupas para calcular la media, mediana y varianza?
\item Calcula los valores
\end{itemize}

Decides comparar esta muestra con la población total, de manera que los datos ahora son: 2500, 2000, 1800, 3000, 2100, 2600, 1900, 1900, 2900, 2100

\begin{itemize}
\item ¿Qué fórmulas ocupas para calcular la media, mediana y varianza?
\item Calcula los valores
\end{itemize}

¿Cómo explicas las diferencias encontradas en estos resultados?

\miquestion \textbf{Datos no agrupados} Un amigo que trabaja en el aeropuerto observa que estás realizando un estudio acerca de los precios de vuelos redondos a una playa, por lo que te proporciona la distribución poblacional discreta que han sacado de manera oficial:

\begin{center}
\begin{tabular}{ |c|c| } 
 \hline
 \textbf{x} & \textbf{P(x)} \\ 
 \hline
 1800 & 0.1 \\
 1900 & 0.2 \\
 2000 & 0.1 \\ 
 2100 & 0.2 \\ 
 2500 & 0.1 \\ 
 2600 & 0.1 \\ 
 2900 & 0.1 \\ 
 3000 & 0.1 \\ 
 \hline
\end{tabular}
\end{center}

\begin{itemize}
\item ¿Qué fórmulas ocupas para calcular la media y varianza? Adicional, explica como encuentras la mediana.
\item Calcula los valores
\end{itemize}

\miquestion \textbf{Función de probabilidad discreta.} Sea X una variable aleatoria con función de probabilidad:

\begin{center}
\begin{tabular}{ |c|c| } 
 \hline
 \textbf{x} & \textbf{P(x)} \\ 
 \hline
 0 & 2c \\ 
 1 & 0.05 \\ 
 2 & 0.2 \\ 
 3 & 0.4 \\ 
 4 & 0.2 \\ 
 5 & c \\ 
 \hline
\end{tabular}
\end{center}

\begin{itemize}
\item Determine el valor de c para que sea una función de probabilidad
\item En este particular ejemplo, utilizarías fórmulas muestrales o poblacionales. Justifica
\item Determina $E(x)$, $Var(x)$ y el coeficiente de variación
\item Obtenga $E(30x-2)$, $Var(30x-2)$ (suponga iid)
\item Obtenga la función de distribución de probabilidad acumulada y grafique
\end{itemize}

\miquestion \textbf{Distribución de probabilidad continua.} Sea X una variable aleatoria que denota el tiempo en minutos que una persona tiene que esperar hasta que pasa el camion en cierto lugar de la ciudad. Suponga que x es uniforme en el intervalo (0,100), es decir que la función de densidad es:

\begin{itemize}
\item Obtenga la función de distribución acumulada de X
\item Calcule el valor esperado, la desviación estándar y la mediana de X
\item Una persona se dirige a tomar el camión. ¿Cuál es la probabilidad que tenga que esperar más de 10 minutos?
\end{itemize}



\end{questions}

\end{document}