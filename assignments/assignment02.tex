%Example of use of oxmathproblems latex class for problem sheets
\documentclass{oxmathproblems}
\usepackage{blindtext}
\usepackage{hyperref}
%(un)comment this line to enable/disable output of any solutions in the file
%\printanswers

%define the page header/title info
\course{ITAM - Estadistica 2}
\oxfordterm{Repaso Estadistica 1 (parte 2/2)}
\sheetnumber{1}
\sheettitle{}

\begin{document}


\begin{questions}

\miquestion \textbf{Correlación entre dos variables aleatorias} 

El departamento de mercadotecnia de una empresa desea investigar la correlación que existe entre las variables X: inversión en una campaña promocional y Y: incremento en ventas después de la promoción.

La función de probabilidad conjunta para las variables anteriores se presentan a continuación (X en renglones, Y en columnas)

\begin{center}
\begin{tabular}{ |c|c|c|c|c|c| } 
 \hline
 \textbf{x/y} & \textbf{0.01} & \textbf{2.0} & \textbf{3.6} & \textbf{3.9} & \textbf{P(X = x)} \\
 \hline
 0.1 & 0.098 & 0.100 & 0.010 & 0.010 & 0.218 \\
 2.5 & 0.001 & 0.010 & 0.200 & 0.001 & 0.212 \\
 2.8 & 0.001 & 0.099 & 0.200 & 0.270 & 0.570 \\ 
 \hline
 P(Y = y) & 0.100 & 0.209 & 0.410 & 0.281 & 1.000 \\
 \hline
\end{tabular}
\end{center}

\begin{itemize}
\item Calcule E(X), E(Y), Var(X), Var(Y), E(XY), Cov(XY)
\item Calcula el coeficiente de correlación de X,Y
\item Interprete
\end{itemize}

\miquestion \textbf{Valores en tablas} 
Usando las tablas oficiales calcule los siguientes valores. Recuerde que en distribuciones continuas no hay diferencia si expresamos $P(Z\leq 3)$ y $P(Z < 3)$, aunque por convención, lo expresaremos como: $P(Z\leq 3)$. Para distribuciones discretas tener cuidado ya que SI hay diferencia entre $P(Y\leq 3)$ y $P(Y < 3)$.


Usando tablas de Normal encuentre las siguientes probabilidades:
\begin{itemize}
\item $P(-2 \leq Z \leq 2)$
\item $P(Z > 3)$

Usando tablas de Normal encuentre el valor de a:
\item $P(Z > a) = 0.5$
\item $P(-a \leq Z \leq a) = 0.98$
\end{itemize}

Usando tablas de Binomial encuentre las siguientes probabilidades:
\begin{itemize}
\item $n = 5, p = 0.5$, $P(Y \leq 2)$
\item $n = 10, p = 0.3$ $P(Y > 5)$

Usando tablas de Binomial encuentre el valor de a:
\item $n = 5, p = 0.5$, $P(Y > a) = 0.5$
\item $n = 10, p = 0.3$, $P(Y < a) = 0.998$
\end{itemize}

Usando tablas de Poisson encuentre las siguientes probabilidades:
\begin{itemize}
\item $\lambda = 3$, $P(X \leq 3)$
\item $\lambda = 3$, $P(1 < X < 3)$

Usando tablas de Poisson encuentre el valor de a:
\item $\lambda = 3$, $P(X > a) = 0.353$
\item $\lambda = 3$, $P(X \leq a) = 0.996$
\end{itemize}


\end{questions}

\end{document}