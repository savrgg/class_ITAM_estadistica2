%%% Template originaly created by Karol Kozioł (mail@karol-koziol.net) and modified for ShareLaTeX use

\documentclass[addpoints]{exam}

\usepackage[T1]{fontenc}
\usepackage[utf8]{inputenc}
\usepackage{graphicx}
\usepackage{xcolor}

\renewcommand\familydefault{\sfdefault}
\renewcommand{\theenumi}{\Alph{enumi}}
\usepackage{tgheros}

\usepackage{amsmath}
\usepackage{amssymb,amsthm,textcomp}
\usepackage{enumerate}
\usepackage{multicol}
\usepackage{tikz}
\usepackage[spanish, es-nodecimaldot]{babel}
\usepackage{enumitem}


\usepackage{geometry}
\geometry{left=25mm,right=25mm,bindingoffset=0mm, top=20mm,bottom=20mm}


\linespread{1.3}

\newcommand{\linia}{\rule{\linewidth}{0.5pt}}

% custom theorems if needed
\newtheoremstyle{mytheor}
{1ex}{1ex}{\normalfont}{0pt}{\scshape}{.}{1ex}
{{\thmname{#1 }}{\thmnumber{#2}}{\thmnote{ (#3)}}}
  
  \theoremstyle{mytheor}
  \newtheorem{defi}{Definition}
  
  % my own titles
  \makeatletter
  \renewcommand{\maketitle}{
    \begin{center}
    \vspace{2ex}
    {\huge \textsc{\@title}}
    \vspace{1ex}
    \\
    \linia\\
    \@author \hfill \@date
    \vspace{4ex}
    \end{center}
  }
  \makeatother
  %%%
  
  % custom footers and headers
  %\usepackage{fancyhdr}
  %\pagestyle{fancy}
  \lfoot{Parcial \textnumero{} 1}
  \cfoot{}
  \rfoot{Page \thepage}
  %\renewcommand{\headrulewidth}{0pt}
  %\renewcommand{\footrulewidth}{0pt}
  
  % code listing settings
  \usepackage{listings}
  \lstset{
    language=Python,
    basicstyle=\ttfamily\small,
    aboveskip={1.0\baselineskip},
    belowskip={1.0\baselineskip},
    columns=fixed,
    extendedchars=true,
    breaklines=true,
    tabsize=4,
    prebreak=\raisebox{0ex}[0ex][0ex]{\ensuremath{\hookleftarrow}},
    frame=lines,
    showtabs=false,
    showspaces=false,
    showstringspaces=false,
    keywordstyle=\color[rgb]{0.627,0.126,0.941},
    commentstyle=\color[rgb]{0.133,0.545,0.133},
    stringstyle=\color[rgb]{01,0,0},
    numbers=left,
    numberstyle=\small,
    stepnumber=1,
    numbersep=10pt,
    captionpos=t,
    escapeinside={\%*}{*)}
  }
  %%%----------%%%----------%%%----------%%%----------%%%
  
  \begin{document}
  
  \title{Parcial 1 - Estadística II}
  
  \author{ITAM, Primavera 2022}
  
  \date{16/03/2022}
  
  \maketitle
  
  \section*{Instrucciones}
  
  El examen cuenta con 10 preguntas a desarrollar. La formalidad al escribir los resultados es considerada como parte del puntaje. En caso de no tener el desarrollo de la pregunta, o bien se llegué a la respuesta sin una justificación se anulará la respuesta. Cualquier práctica fraudulenta será sancionada de acuerdo al reglamento del departamento. \textbf{Trabajar con 4 cifras decimales}

\vspace{10pt}

\begin{questions} 

\section*{Seccion A: Estimación puntual}

\question \textbf{(10pts)} Se toma una muestra del número de matches que se tienen en una app para conocer personas llamada $Bubble$ durante 5 dias aleatorios. Se obtienen los siguientes datos:

$$75, 92, 517, 3200, 428$$

\begin{enumerate}[label=\Alph*)]
\item Determina el número promedio de matches por dia
\item La varianza del número de matches por dia
\item La proporción de dias que cuentan con más de 100 matches al dia
\end{enumerate}

\section*{Seccion B: Distribuciones de muestreo exactas}

\question \textbf{(15pts)} Suponga que una variable aleatoria X puede tomar los valores \{10,50,70\} con probabilidades \{0.1, 0.3, 0.6\}. Considere muestras de tamaño 2 con reemplazo.
  
  \begin{enumerate}
  \item (3 pts) Calcule $E(X)$ y $V(X)$
  \item (3 pts) Obtenga el espacio muestral y sus respectivas probabilidades
  \item (3 pts) Obtenga distribución de muestreo de $\overline{X}$, $E(\overline{X})$ y $V(\overline{X})$ ¿Qué puede concluir comparandola con lo obtenido en A)?
  \item (1 pts) ¿La distribución de $\overline{X}$ es exacta? Justifique
  \end{enumerate}

 \question \textbf{(10 pts)} En una fabrica de galletas tienen 5 lineas de producción $L_1$, $L_2$, $L_3$, $L_4$, $L_5$. Se sabe que que las lineas $L_3$, $L_4$, $L_5$ requieren reparación.

  \begin{enumerate}
  \item (5 pts) Obtenga el espacio muestral si se toman muestras de dos lineas sin reemplazo
  \item (5 pts) Obtenga la distribución de muestreo de la proporción de lineas $\hat{p}$ que necesitan reparación
  \end{enumerate}

\section*{Seccion C: Distribuciones de muestreo aproximada}

\question \textbf{(10pts)} 
Se decide hacer un evento para apoyar monetariamente a refugios de animales sin hogar. Se sabe que el monto que dona cada persona (X) es una variable aleatoria con media 300 y desviación estándar 50. Si en total acuden 100 personas al evento:

\begin{enumerate}[label=\Alph*)]
\item Determine cuál es la probabilidad que se recauden más de $\$30,500$ pesos durante el evento.
\item Si los costos logísticos del evento ascienden a $\$15,000$, ¿Cuál es la probabilidad que se incurran en pérdidas?
\end{enumerate}

\question \textbf{(10pts)} El precio de los armadillos en el zoológico se distribuye normal con media 5 kg y varianza 1.

\begin{enumerate}[label=\Alph*)]
\item ¿Cuál es la probabilidad que un armadillo seleccionado al azar pese más de 4.5 kg?
\item Se toma una muestra de 5 armadillos ¿Cuál es la probabilidad que la varianza de la muestra sea mayor a 1.2?
\end{enumerate}
  

\section*{Seccion D: Propiedades de estimadores}

\question \textbf{(15pts)} Sea $\hat{\mu}$ un estimador a la media poblacional. Resuelva los siguientes incisos: 
   
$$\hat{\mu} = \frac{1}{2} (X_1) + \frac{X_3+...+X_{n-3}}{2b} $$

\begin{enumerate}[label=\Alph*)]
\item Encuentre el ECM del estimador (poner en términos de b)
\item Determina $b$ tal que el estimador sea insesgado
\end{enumerate}

\question \textbf{(15pts)} Sea $Y$ la variable aleatoria que modela el número de alumnos en el ITAM que deciden cambiarse a la carrera de Matemáticas Aplicadas después de llevar Estadística 2. Se sabe que $Y$ se distribuye Poisson con media $\lambda$. Se considera el siguiente estimador a la media:

$$\hat{\lambda} = \sum_i Y_i(Y_i-1) $$

Toma una muestra de tamaño $n$

\begin{enumerate}[label=\Alph*)]
\item Determina si el estimador es insesgado
\end{enumerate}


\question \textbf{(15pts)} La probabilidad de que un alumno se vuelva fit después de una plática de Barby Regil es $p$. Se sabe que $n$ alumnos acudieron a la platica impartida y se propone el siguiente estimador a la proporción poblacional:

$$\hat{p} = \frac{Y+1}{n+1}$$

Sea Y el número de alumnos que se vuelven fit. 

\begin{enumerate}[label=\Alph*)]
\item Determina el sesgo de $\hat{p}$
\end{enumerate}


  \end{questions}
  \end{document}
  