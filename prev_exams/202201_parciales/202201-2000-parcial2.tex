%%% Template originaly created by Karol Kozioł (mail@karol-koziol.net) and modified for ShareLaTeX use

\documentclass[addpoints]{exam}

\usepackage[T1]{fontenc}
\usepackage[utf8]{inputenc}
\usepackage{graphicx}
\usepackage{xcolor}

\renewcommand\familydefault{\sfdefault}
\usepackage{tgheros}

\usepackage{amsmath,amssymb,amsthm,textcomp}
\usepackage{enumerate}
\usepackage{multicol}
\usepackage{tikz}
\usepackage[spanish]{babel}
\usepackage{enumitem}

\decimalpoint

\usepackage{geometry}
\geometry{left=25mm,right=25mm,%
bindingoffset=0mm, top=20mm,bottom=20mm}


\linespread{1.3}

\newcommand{\linia}{\rule{\linewidth}{0.5pt}}

% custom theorems if needed
\newtheoremstyle{mytheor}
{1ex}{1ex}{\normalfont}{0pt}{\scshape}{.}{1ex}
{{\thmname{#1 }}{\thmnumber{#2}}{\thmnote{ (#3)}}}
  
  \theoremstyle{mytheor}
  \newtheorem{defi}{Definition}
  
  % my own titles
  \makeatletter
  \renewcommand{\maketitle}{
    \begin{center}
    \vspace{2ex}
    {\huge \textsc{\@title}}
    \vspace{1ex}
    \\
    \linia\\
    \@author \hfill \@date
    \vspace{4ex}
    \end{center}
  }
  \makeatother
  %%%
  
  % custom footers and headers
  \lfoot{Assignment \textnumero{} 5}
  \cfoot{}
  \rfoot{Page \thepage}
  \renewcommand{\theenumi}{\Alph{enumi}}
  %
  
  % code listing settings
  \usepackage{listings}
  \lstset{
    language=Python,
    basicstyle=\ttfamily\small,
    aboveskip={1.0\baselineskip},
    belowskip={1.0\baselineskip},
    columns=fixed,
    extendedchars=true,
    breaklines=true,
    tabsize=4,
    prebreak=\raisebox{0ex}[0ex][0ex]{\ensuremath{\hookleftarrow}},
    frame=lines,
    showtabs=false,
    showspaces=false,
    showstringspaces=false,
    keywordstyle=\color[rgb]{0.627,0.126,0.941},
    commentstyle=\color[rgb]{0.133,0.545,0.133},
    stringstyle=\color[rgb]{01,0,0},
    numbers=left,
    numberstyle=\small,
    stepnumber=1,
    numbersep=10pt,
    captionpos=t,
    escapeinside={\%*}{*)}
  }
  
  %%%----------%%%----------%%%----------%%%----------%%%
  
  \begin{document}
  
  \title{Parcial 2 - Estadística II 20:00}
  
  \author{ITAM, Primavera 2022}
  
  \date{09/05/2022}
  
  \maketitle
  
  \section*{Instrucciones}
 

\vspace{10pt}
  
El examen tiene una duración de 1 hora. \textbf{Cualquier práctica fraudulenta será sancionada de acuerdo al reglamento.} La hora de entrega es 22:00 

\section*{Seccion A: Preguntas a desarrollar (100 pts)}
  
  
\begin{questions} 

\question \textbf{(10pts, en casa)}  La cafetería del ITAM sigue un proceso de calidad en sus alimentos. Se sabe que la proporción de alimentos en mal estado debe ser máximo de $14\%$. El ITAM toma como muestra 30 alimentos y decide que si encuentra más de 8 alimentos en mal estado deberá cerrar la cafeteria. Usted es contratado para ayudar al ITAM a contestar las siguientes preguntas: (use $\alpha = .05$). 
  
\begin{enumerate}[label=\Alph*)]
\item (2pts) Enuncie las hipótesis nula y alternativa
\item (3pts) Obtenga la probabilidad de cometer el error tipo I y tipo II
\item (5pts) Obtenga la potencia de la prueba cuando la proporción de alimentos defectuosas es del $20\%$ y $30\%$. 
\end{enumerate}
 
 
   \question \textbf{(10pts)} En tiempos de Covid, se desea saber si la temperatura promedio de los empleados de un grupo empresarial es distinta que en tiempos normales, para esto se recolectó una muestra antes y una durante la pandemia, obteniendose los siguientes datos (use $\alpha = .05$):

    \begin{table}[h]
    \centering
    \begin{tabular}{ll}
    Durante Covid19 & Pre-Covid19 \\
    $n_1 = 20$ & $n_2 = 20$ \\
    $\bar{y_1} =  78$ & $\bar{y_2} = 67$ \\
    $s_1 = 22$ &  $s_2 = 20$
    \end{tabular}
    \end{table}

\begin{enumerate}[label=\Alph*)]
  \item (7pts) ¿Hay suficiente evidencia para decir que existe una diferencia en la temperatura promedio de los empleados del grupo? Concluye.
  \item (3pts) ¿Cuál es el nivel de significancia alcanzado (valor-p)?
  \end{enumerate}
  
  
 \question \textbf{(15pts)} Las vaquitas marinas son cetáceos que les gusta buscar calamares y peces cerca de aguas poco profundas. Un grupo de investigadores está interesado en medir las libras promedio que consumen por semana, la cual se distribuye normal. Los investigadores suponen que consumen $800$ libras (o menos) con $\sigma$ menor que 40, bastante abajo del límite que plantea la revista Nature, que es de $1000$ libras semanales. Con esto deciden tomar una muestra de $n = 40$ vaquitas, encontrando que la media muestra y la varianza son igual a $825$ libras y $2350$ libras respectivamente (use $\alpha = 0.05$).
  
\begin{enumerate}[label=\Alph*)]
  \item  (3pts) si $\mu = 800$ y $\sigma = 40$, ¿Que tan probable es que es una vaquita marina consuma más de 1000 libras a la semana?
  \item (6pts) ¿Los datos proporcionan suficiente evidencia para indicar que las vaquitas marinas consumen más de 800 libras? Plantea $H_0$, $H_1$, usa TCL y concluye (no use la varianza del inciso a), si no la varianza muestral del enunciado del ejercicio). Encuentre el valorp
  \item (6pts) ¿Los datos aportan suficiente evidencia para indicar $\sigma$ excede de 40? Plantea $H_0$ y $H_1$ y concluye. Encuentre el valor p
  \end{enumerate}
  
 
 \question \textbf{(10 pts)} El Gobierno de la Ciudad de México decide comprar una cantidad importante de semáforos. El productor de estos semáforos afirma que tienen una vida promedio de 5 años, con una varianza menor o igual a 2 años. El gobierno decide adquirirlos solamente si tienen una varianza menor o igual 2 años de duración. Se seleccionan al azar 30 semáforos y se obtiene $S^2 = 3.1$. Usted forma parte del gobierno de la Ciudad de México y piensa que lo dicho por el productor es mentira, por lo que busca contrastar lo afirmado por él.

\begin{enumerate}
\item Formule las hipótesis $H_0$ y $H_1$ apropiadas al problemas
\item Pruebe con un nivel de significancia de $\alpha = 0.05$
\item Muestre en una gráfica la región de rechazo y el estadístico de prueba correspondiente
\item Inteprete el resultado
\end{enumerate}

\question \textbf{(10 pts)}
La dirección del ITAM desea saber si existe una diferencia del doble de la varianza de las calificaciones de las clases en linea comparándolas con las clases presenciales ($2*\sigma^2_{Presencial} = \sigma^2_{Online}$). Para esto selecciona las calificaciones que tuvieron 11 alumnos previo a la pandemia (X: clases presenciales), contra 16 en la situación actual (Y: clase en linea). Los datos obtenidos son los siguientes: $\bar{X} = 6.92$, $\bar{Y} = 9.48$, $S_{X} = 1.46$, $S_{Y} = 1.11$

\begin{enumerate}
\item Formule las hipótesis $H_0$ y $H_1$ apropiadas al problemas
\item Pruebe con un nivel de significancia de $\alpha = 0.05$
\item Muestre en una gráfica la región de rechazo y el estadístico de prueba correspondiente
\item Inteprete el resultado
\end{enumerate}


\end{questions}
\end{document}
  

  