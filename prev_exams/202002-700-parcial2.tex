%%% Template originaly created by Karol Kozioł (mail@karol-koziol.net) and modified for ShareLaTeX use

\documentclass[addpoints]{exam}

\usepackage[T1]{fontenc}
\usepackage[utf8]{inputenc}
\usepackage{graphicx}
\usepackage{xcolor}

\renewcommand\familydefault{\sfdefault}
\usepackage{tgheros}

\usepackage{amsmath,amssymb,amsthm,textcomp}
\usepackage{enumerate}
\usepackage{multicol}
\usepackage{tikz}
\usepackage[spanish]{babel}
\usepackage{enumitem}

\usepackage{geometry}
\geometry{left=25mm,right=25mm,%
bindingoffset=0mm, top=20mm,bottom=20mm}


\linespread{1.3}

\newcommand{\linia}{\rule{\linewidth}{0.5pt}}

% custom theorems if needed
\newtheoremstyle{mytheor}
{1ex}{1ex}{\normalfont}{0pt}{\scshape}{.}{1ex}
{{\thmname{#1 }}{\thmnumber{#2}}{\thmnote{ (#3)}}}
  
  \theoremstyle{mytheor}
  \newtheorem{defi}{Definition}
  
  % my own titles
  \makeatletter
  \renewcommand{\maketitle}{
    \begin{center}
    \vspace{2ex}
    {\huge \textsc{\@title}}
    \vspace{1ex}
    \\
    \linia\\
    \@author \hfill \@date
    \vspace{4ex}
    \end{center}
  }
  \makeatother
  %%%
  
  % custom footers and header
  \lfoot{Assignment \textnumero{} 5}
  \cfoot{}
  \rfoot{Page \thepage}
  \renewcommand{\theenumi}{\Alph{enumi}}
  %
  
  % code listing settings
  \usepackage{listings}
  \lstset{
    language=Python,
    basicstyle=\ttfamily\small,
    aboveskip={1.0\baselineskip},
    belowskip={1.0\baselineskip},
    columns=fixed,
    extendedchars=true,
    breaklines=true,
    tabsize=4,
    prebreak=\raisebox{0ex}[0ex][0ex]{\ensuremath{\hookleftarrow}},
    frame=lines,
    showtabs=false,
    showspaces=false,
    showstringspaces=false,
    keywordstyle=\color[rgb]{0.627,0.126,0.941},
    commentstyle=\color[rgb]{0.133,0.545,0.133},
    stringstyle=\color[rgb]{01,0,0},
    numbers=left,
    numberstyle=\small,
    stepnumber=1,
    numbersep=10pt,
    captionpos=t,
    escapeinside={\%*}{*)}
  }
  
  %%%----------%%%----------%%%----------%%%----------%%%
  
  \begin{document}
  
  \title{Parcial 1 - Estadística II}
  
  \author{ITAM, Primavera 2020}
  
  \date{13/10/2020}
  
  \maketitle
  
  \section*{Instrucciones}
  
 El examen consta de una sección. Se deberán desarrollar todos los problema planteados. La formalidad y correctez al escribir los resultados será evaluada. En caso de no tener el desarrollo de la pregunta, o bien se llegue a la respuesta sin una justificación la respuesta será anulada. 
  
  \vspace{10pt}
  
El examen tiene una duración de 1:45 horas. Cualquier práctica fraudulenta será sancionada de acuerdo al reglamento del departamento. Se debe entregar el examen a más tardar a las 8:50 en el siguiente correo: salvador.garcia.gonzalez@itam.mx

  \section*{Seccion A: Preguntas a desarrollar (100 pts)}
 
  \begin{questions} 
  \question (15 pts) Determine el error cuadrático medio del siguiente estimador:
  
  $$\hat{\mu} = \frac{1}{8} (X_1) + \frac{X_3+...+X_{n-3}}{2(n-5)} + \frac{1}{8} (X_{n-2}+X_{n-1}+X_{n}) $$
  
  \question (15 pts) Un instrumento para medir la cantidad un químico sanguíneo está garantizado para dar lecturas que no varían más de 2 unidades. Una muestra de 4 lecturas del instrumento en el mismo objeto proporcionó mediciones de 353, 351, 351, 355. 
  
  \begin{enumerate}
  \item Encuentre el intervalo de confianza del 90\% para la varianza poblacional.
  \item Concluya si la muestra respalda la garantía del instrumento.
  \end{enumerate}
   
  
  \question (15 pts) Determine el sesgo de la \textit{pooled variance} $S_p^2 = \frac{(n_1-1) S_1^2 + (n_2-1) S_2^2}{n_1 + n_2 -2}$. Suponga que $S_1^2$ y $S_2^2$ son los estimadores insesgados de la varianza. (Suponer que ambas poblaciones tienen la misma varianza)

  \question (10 pts) Explique la relación entre B = longitud del intervalo de confianza, el nivel de confianza $(1-\alpha)$ y el tamaño de muestra $n$.
  
  \question (15 pts) Una aplicación de viajes compartidos desea estimar el tiempo promedio de los viajes. Se selecciona aleatoriamente 16 viajes y los resultados son los siguientes (en minutos):
  
  $$\sum X_i = 336$$ $$ \sum X_i^2 = 7,116$$
  
  \begin{enumerate}
\item Determine un intervalo de confianza del 99\% para el tiempo promedio de viaje
  \end{enumerate}
  
  \question (15 pts) Un pastor desea estimar la altura promedio de sus ovejas. Se conoce a priori que la altura de las ovejas tiene una distribución normal con desviación estándar de 4 centímetros. 
  \begin{enumerate}
  \item Si se selecciona una muestra de 9 ovejas, encuentra la probabilidad que la media muestral difiera a lo más 2 centímetros de la media poblacional.
  \item Suponga que el pastor quiere que la media de la muestra difiera a lo más 1 centímetro de la media de la población con una probabilidad de .90 ¿Cuántas ovejas tendría que medir para estar seguro de obtener ese grado de exactitud?
  \end{enumerate}
  
  

\question (15 pts) Un intervalo de confianza es insesgado si el valor esperado del punto medio del intervalo es igual al parámetro estimado. 
\begin{enumerate}
\item Determine si el intervalo de confianza para $\sigma^2$ es insesgado.
\end{enumerate}

\end{questions}
  
\end{document}
  