%%% Template originaly created by Karol Kozioł (mail@karol-koziol.net) and modified for ShareLaTeX use

\documentclass[addpoints]{exam}

\usepackage[T1]{fontenc}
\usepackage[utf8]{inputenc}
\usepackage{graphicx}
\usepackage{xcolor}

\renewcommand\familydefault{\sfdefault}
\usepackage{tgheros}

\usepackage{amsmath}
\usepackage{amssymb,amsthm,textcomp}
\usepackage{enumerate}
\usepackage{multicol}
\usepackage{tikz}
\usepackage[spanish]{babel}

\usepackage{geometry}
\geometry{left=25mm,right=25mm,bindingoffset=0mm, top=20mm,bottom=20mm}


\linespread{1.3}

\newcommand{\linia}{\rule{\linewidth}{0.5pt}}

% custom theorems if needed
\newtheoremstyle{mytheor}
{1ex}{1ex}{\normalfont}{0pt}{\scshape}{.}{1ex}
{{\thmname{#1 }}{\thmnumber{#2}}{\thmnote{ (#3)}}}
  
  \theoremstyle{mytheor}
  \newtheorem{defi}{Definition}
  
  % my own titles
  \makeatletter
  \renewcommand{\maketitle}{
    \begin{center}
    \vspace{2ex}
    {\huge \textsc{\@title}}
    \vspace{1ex}
    \\
    \linia\\
    \@author \hfill \@date
    \vspace{4ex}
    \end{center}
  }
  \makeatother
  %%%
  
  % custom footers and headers
  %\usepackage{fancyhdr}
  %\pagestyle{fancy}
  \lfoot{Assignment \textnumero{} 5}
  \cfoot{}
  \rfoot{Page \thepage}
  %\renewcommand{\headrulewidth}{0pt}
  %\renewcommand{\footrulewidth}{0pt}
  
  % code listing settings
  \usepackage{listings}
  \lstset{
    language=Python,
    basicstyle=\ttfamily\small,
    aboveskip={1.0\baselineskip},
    belowskip={1.0\baselineskip},
    columns=fixed,
    extendedchars=true,
    breaklines=true,
    tabsize=4,
    prebreak=\raisebox{0ex}[0ex][0ex]{\ensuremath{\hookleftarrow}},
    frame=lines,
    showtabs=false,
    showspaces=false,
    showstringspaces=false,
    keywordstyle=\color[rgb]{0.627,0.126,0.941},
    commentstyle=\color[rgb]{0.133,0.545,0.133},
    stringstyle=\color[rgb]{01,0,0},
    numbers=left,
    numberstyle=\small,
    stepnumber=1,
    numbersep=10pt,
    captionpos=t,
    escapeinside={\%*}{*)}
  }
  
  %%%----------%%%----------%%%----------%%%----------%%%
  
  \begin{document}
  
  \title{Parcial 1 - Estadística II}
  
  \author{ITAM, Otoño 2021}
  
  \date{05/10/2021}
  
  \maketitle
  
  \section*{Instrucciones}
  
  
El examen tiene una duración estricta de 1:40 horas y comienza a las 20:00 hrs. Por cada minuto de retraso se descontarán 5 pts por lo que se debe cuidar mucho el tiempo. Pasadas las 21:45 horas no se recibirán más exámenes. Se seleccionarán 5 exámenes aleatoriamente para justificar alguna respuesta de su examen en los últimos 15 minutos de clase, por lo que es necesario permanecer en la sala de zoom durante las dos horas. No es necesario contar con cámaras encendidas.\textbf{Cualquier práctica fraudulenta será notificada al departamento de estadística y será sancionada de acuerdo al reglamento.} 

El examen consta de dos secciones. La primera es opción múltiple y V/F donde se deberá seleccionar la opción correcta. Para cada pregunta hay una sola respuesta correcta. Si se selecciona más de una opción será considerada como incorrecta. En la segunda sección se deberá desarrollar el problema planteado. Se debe cuidar la formalidad al escribir los resultados, ya que es parte de la calificación del problema. En caso de no tener el desarrollo de la pregunta, o bien se llegué a la respuesta sin una justificación se podrá anular la respuesta. 
  
  \vspace{10pt}

  
  \section*{Seccion A: Opción múltiple y V/F(30 pts)}
  
  \begin{questions}
  
  \question (8 pts) MUESTREO: Un fabricante de Estados Unidos produce autos que garantizan que la pintura dura al menos 5 años sin corrosión. Cómo parte del proceso de calidad se decide tomar una muestra aleatoria de 100 autos con más de 5 años para ver el estado de la puntura (sin corrosión / con corrosión).

Los automóviles producidos por la fábrica desde el dia 1 es:

\begin{checkboxes}
  \choice Marcos muestrales
  \choice Unidades de muestreo
  \choice Población
  \choice Muestra
  \choice Ninguna de las anteriores
\end{checkboxes}
  
Los automóviles con más de 5 años producidos por la fábrica es:

\begin{checkboxes}
  \choice Marcos muestrales
  \choice Variable de interes
  \choice Población
  \choice Muestra
  \choice Ninguna de las anteriores
\end{checkboxes}


Cada uno de los automóviles con más de 5 años es:
\begin{checkboxes}
  \choice Muestra
  \choice Unidades de muestreo
  \choice Población
  \choice Variable de interés
  \choice Ninguna de las anteriores
\end{checkboxes}

El conjunto de los 100 autos es:
\begin{checkboxes}
  \choice Muestra
  \choice Unidades de muestreo
  \choice Población
  \choice Variable de interés
  \choice Ninguna de las anteriores
\end{checkboxes}

\question (2 pts) MUESTREO: Son colecciones no traslapadas de elementos de la población que cubren la población completa.
  
  \begin{checkboxes}
  \choice Marcos muestrales
  \choice Unidades de muestreo
  \choice Elemento muestral
  \choice Muestra
  \end{checkboxes}
  
\question (2 pts) MUESTREO: Es una lista de unidades de muestreo.
  
  \begin{checkboxes}
  \choice Marcos muestrales
  \choice Unidades de muestreo
  \choice Elemento muestral
  \choice Muestra
  \end{checkboxes}
  
  \question (2 pts) MUESTREO:  Consiste en seleccionar una muestra mediante la separación de los elementos de la población en grupos que no presenten traslapes, de manera que cada grupo sea homogéneo entre sus elementos, pero heterogéneo entre grupos.
  
  \begin{checkboxes}
  \choice Muestreo aleatorio simple
  \choice Muestreo aleatorio estratificado
  \choice Muestreo por conglomerados
  \choice Muestreo por cuotas
  \end{checkboxes}

\question (2 pts) DISTRIBUCIONES MUESTRALES: Sean $Y_1$, $Y_2$, ..., $Y_n$ variables iid distribuidas normalmente con $E(Y_i) = \mu_i$ y $V(Y_i) = \sigma_i^2$ para i = 1,2,...,n. Definiendo $Z_i = \frac{Y_i - \mu_i}{\sigma_i}$, entonces $\sum_{i = 1}^{n} Z_i^2$ sigue una distribución:
    \begin{checkboxes}
  \choice t-student con $n-1$ g.l.
  \choice Normal con $\mu = 0$ y $\sigma = 1$ 
    \choice F con $n-1$ y $m-1$ g.l.
  \choice $\chi^2$ con $n-1$ g.l.
  \end{checkboxes}
  
  
  \question (2 pts) DISTRIBUCIONES MUESTRALES: Sean $W_1$ y $W_2$ v.a. independientes con distribución $\chi_2$ con $v_1$ y $v_2$ g.l. respectivamente. Entonces se dice que: $\frac{W_1/v_1}{W_2/v_2}$ sigue una distribución:
    
    \begin{checkboxes}
  \choice t-student con $v_1$ g.l.
  \choice Normal con $\mu = 0$ y $\sigma = 1$ 
    \choice F con $v_1$ y $v_2$ g.l.
  \choice $\chi^2$ con $v_1$ g.l.
  \end{checkboxes}
  
  \question (2 pts) DISTRIBUCIONES MUESTRALES: Es una función de las variables aleatorias observables en una muestra y de constantes conocidas.
  
  \begin{checkboxes}
  \choice Estadístico 
  \choice Parámetro
  \choice Distribución
  \choice Media poblacional
  \end{checkboxes}
  
  \question (10 pts) Determine si las siguientes afirmaciones son verdaderas o falsas.
  
  \begin{enumerate}
  \item El objetivo de la inferencia estadística es analizar a toda la población para obtener conclusiones
  \item Si se toman dos muestras aleatorias del mismo tamaño de la misma población, el valor del estadístico que se calcule será el mismo para ambas
  \item En una distribución de muestreo exacta, buscamos tener una sola muestra y hacer inferencia estadística con ella. 
  \item La varianza de la media muestral para un tamaño de muestra n>1, siempre es menor que la varianza de la población completa (con reemplazo)
  \item Se tiene el experimento de lanzar una moneda y donde los posibles valores son {aguila, sol}. Se toma una muestra de tamaño 1, si se calcula la distribución de muestreo exacta para $\bar{X}$ es la misma que la de la distribución original
  \end{enumerate}
  
  
\end{questions}
  
  \section*{Seccion B: Preguntas a desarrollar (70 pts)}
  
  \begin{questions} 

  \question (20 pts)
  Para las siguientes preguntas resuelve 
  \begin{enumerate}
  \item (6 pts) El peso de unas cajas se distribuye de forma normal con media 10 kgs y desviación estándar de 3. ¿Si se toma una muestra de 9 productos, ¿Cuál es la probabilidad que la varianza  muestral esté entre 1 y 3?
  \item (7 pts) Se sabe que la demanda de transporte en una plataforma digital en una región se distribuye Poisson con media de 60 por dia. En otra plataforma se distribuye Poisson con media de 50 por dia.Calcule la probabilidad de que diferencia de medias muestrales entre ambas plataformas sea mayor a 20
  \item (7 pts) En una caja, las variaciones en el número personas atendidas por los cajeros A y B es la misma. Se obtuvo el número promedio de personas por dia por cada cajero, así como la varianza muestral. Esto se realizó durante 16 y 21 dias respectivamente. ¿Cuál es la probabilidad que el cociente de varianzas muestrales exceda 0.54 suponiendo que, los números de unidades manejadas por día por los cajeros son v.a. independientes que se distribuyen normal?
  \end{enumerate}

\question (15 pts) Si $\bar{Y_1}$ y $\bar{Y_2}$ son medias muestrales de dos poblaciones con las distribuciones mostradas arriba:

\begin{table}[h!]
\centering
\begin{tabular}{llll}
$Y_1$ & 2 & 5 & 8 \\ \hline
$P(Y_1)$ & 2/3 & 1/6 & 1/6 
\end{tabular}
\end{table}

\begin{table}[h!]
\centering
\begin{tabular}{llll}
$Y_2$ & 1 & 5 \\ \hline
$P(Y_2)$ & 1/2 & 1/2
\end{tabular}
\end{table}


Obtenga la probabilidad de que $\bar{Y_1} + \bar{Y_2}$ sea mayor que 8 cuando se toman muestras aleatorias independientes con reemplazo de tamaño $n_1 = 125$, $n_2 = 100$


 \question (10 pts) En una fabrica de galletas tienen 5 lineas de producción $L_1$, $L_2$, $L_3$, $L_4$, $L_5$. Se sabe que que las lineas $L_3$, $L_4$, $L_5$ requieren reparación.

  \begin{enumerate}
  \item (5 pts) Obtenga el espacio muestral si se toman muestras de dos lineas sin reemplazo
  \item (5 pts) Obtenga la distribución de muestreo de la proporción de lineas $\hat{p}$ que necesitan reparación, 
  \item (5 pts) Calcule el valor esperado de la distribución de muestreo y la varianza de la distribución de muestreo $E(\hat{p})$ y $V(\hat{p})$
  \end{enumerate}
  
  
  \question (10 pts) Suponga que una variable aleatoria X puede tomar los valores \{10,50,70\} con probabilidades \{0.1, 0.3, 0.6\}. Considere muestras de tamaño 2 con reemplazo.
  
  \begin{enumerate}
  \item (3 pts) Calcule $E(X)$ y $V(X)$
  \item (3 pts) Obtenga el espacio muestral y sus respectivas probabilidades
  \item (3 pts) Obtenga distribución de muestreo de $\overline{X}$, $E(\overline{X})$ y $V(\overline{X})$
  \item (1 pts) ¿La distribución de $\overline{X}$ es exacta? Justifique
  \end{enumerate}

 \question (15 pts) Encuentre si el ECM del siguiente estimador. Determine si es insesgado. 
   
$$\hat{\mu} = \frac{1}{8} (X_1) + \frac{X_3+...+X_{n-3}}{2(n-5)} + \frac{1}{8} (X_{n-2}+X_{n-1}+X_{n}) $$


\end{questions}

\end{document}
  