%%% Template originaly created by Karol Kozioł (mail@karol-koziol.net) and modified for ShareLaTeX use

\documentclass[addpoints]{exam}

\usepackage[T1]{fontenc}
\usepackage[utf8]{inputenc}
\usepackage{graphicx}
\usepackage{xcolor}

\renewcommand\familydefault{\sfdefault}
\usepackage{tgheros}

\usepackage{amsmath,amssymb,amsthm,textcomp}
\usepackage{enumerate}
\usepackage{multicol}
\usepackage{tikz}
\usepackage[spanish]{babel}

\usepackage{geometry}
\geometry{left=25mm,right=25mm,%
bindingoffset=0mm, top=20mm,bottom=20mm}


\linespread{1.3}

\newcommand{\linia}{\rule{\linewidth}{0.5pt}}

% custom theorems if needed
\newtheoremstyle{mytheor}
{1ex}{1ex}{\normalfont}{0pt}{\scshape}{.}{1ex}
{{\thmname{#1 }}{\thmnumber{#2}}{\thmnote{ (#3)}}}
  
  \theoremstyle{mytheor}
  \newtheorem{defi}{Definition}
  
  % my own titles
  \makeatletter
  \renewcommand{\maketitle}{
    \begin{center}
    \vspace{2ex}
    {\huge \textsc{\@title}}
    \vspace{1ex}
    \\
    \linia\\
    \@author \hfill \@date
    \vspace{4ex}
    \end{center}
  }
  \makeatother
  %%%
  
  % custom footers and headers
  \lfoot{Assignment \textnumero{} 5}
  \cfoot{}
  \rfoot{Page \thepage}
  %
  
  % code listing settings
  \usepackage{listings}
  \lstset{
    language=Python,
    basicstyle=\ttfamily\small,
    aboveskip={1.0\baselineskip},
    belowskip={1.0\baselineskip},
    columns=fixed,
    extendedchars=true,
    breaklines=true,
    tabsize=4,
    prebreak=\raisebox{0ex}[0ex][0ex]{\ensuremath{\hookleftarrow}},
    frame=lines,
    showtabs=false,
    showspaces=false,
    showstringspaces=false,
    keywordstyle=\color[rgb]{0.627,0.126,0.941},
    commentstyle=\color[rgb]{0.133,0.545,0.133},
    stringstyle=\color[rgb]{01,0,0},
    numbers=left,
    numberstyle=\small,
    stepnumber=1,
    numbersep=10pt,
    captionpos=t,
    escapeinside={\%*}{*)}
  }
  
  %%%----------%%%----------%%%----------%%%----------%%%
  
  \begin{document}
  
  \title{Parcial 1 - Estadística II}
  
  \author{ITAM, Otoño 2020}
  
  \date{08/09/2020}
  
  \maketitle
  
  \section*{Instrucciones}
  

 El examen consta de una sección. Se deberá desarrollar el problema planteado. Se debe cuidar la formalidad al escribir los resultados, ya que es parte de la calificación del problema. En caso de no tener el desarrollo de la pregunta, o bien se llegue a la respuesta sin una justificación se podrá anular la respuesta. 
  
  \vspace{10pt}
  
El examen tiene una duración de 1:45 horas. Cualquier práctica fraudulenta será sancionada de acuerdo al reglamento del departamento. Se tiene hasta las 21:50 para enviar el examen, se debe enviar al siguiente correo: salvador.garcia.gonzalez@itam.mx


  \section*{Seccion A: Preguntas a desarrollar (100 pts)}
  
  
  \begin{questions} 
  \question (20 pts) Suponga que una variable aleatoria puede tomar los valores \{2,4,6\}. Considere muestras de tamaño 2 con reemplazo.
  
  \begin{enumerate}
  \item Calcule $E(X)$ y $V(X)$
  \item Obtenga distribución de muestreo de la mediana  y su respectiva esperanza y varianza
  \item Obtenga distribución de muestreo del siguiente estadístico, $\frac{4}{2}X_1-\frac{1}{2}X_2$ y su respectiva esperanza y varianza
  \item ¿Las distribuciones encontradas en los incisos anteriores son distribuciones aproximadas? Justifique.
  \item ¿Calcule la probabilidad que $3 \leq mediana \leq 5$ ?
  \end{enumerate}
  
  \question (40 pts) 
  La distancia que caminan los alumnos de una clase a otra es una variable aleatoria normal con desviación estándar de 0.01 km. Al tomar una muestra aleatoria de 25 alumnos y medir la distancia que viajan, se obtuvo una desviación estándar de las distancias de 0.015 km. 
  
  \begin{enumerate}
  \item Encuentre la probabilidad que $S_1^2 > (0.015)^2$
  \item Los resultados muestrales, ¿Le hacen dudar de la afirmación de que la varianza de la distancia viajada sea de $\sigma^2 = (0.01)^2$
  \item Si se toma otra muestra aleatoria de 31 alumnos, independiente de la primera. ¿Cuál es la probabilidad de que la varianza de la segunda muestra $S_2^2$ sea menor que dos veces $S_1^2$?
  \end{enumerate}
  
  \question (20 pts) 
  Se sabe que el consumo de cerveza por familia tiene una distribución normal con media desconocida y desviación estándar de 1.25 hectolitros. 
  \begin{enumerate}
  \item Si se toma una muestra de 36 familias y se registra su consumo de cerveza. ¿Cuál es la probabilidad de que la media muestral se encuentre, como máximo, a medio hectolitro de la media de la población?
  \end{enumerate}

  \question (20 pts) 
El número promedio de faltas de los alumnos del ITAM se distribuye exponencial con un promedio de 100 faltas a lo largo de su carrera profesional. Si se toma una muestra aleatoria de 40 egresados del ITAM, ¿Cuál es la probabilidad de que el número de faltas de esta muestra sea mayor de 120 en promedio?
\end{questions}
 
  \section*{Pregunta de rescate (Se deben tener ambas correctas para sumar los 5 puntos, solamente una pregunta correcta no es acreedora a puntos) (5pts)}
  
  \begin{questions}
  \question Suponga que tiene una población con media $\mu$ y varianza $\sigma^2$. ¿Qué distribución sigue $\overline{X}$ suponiendo una muestra de $n = 15$ elementos:

  \begin{checkboxes}
  \choice $N(\mu, \sigma^2)$ y es exacta
  \choice $N(\mu, \sigma^2)$ y es aproximada
  \choice $N(\mu, \frac{\sigma^2}{n})$ y es exacta
  \choice $N(\mu, \frac{\sigma^2}{n})$ y es aproximada
  \choice No es posible saber con los datos
  \end{checkboxes}
    
  \question Suponga que X sigue una distribución exponencial(5). Usando TCL, ¿Qué distribución sigue $\overline{X}$ y $\sum_i X_i$ cuando $n=30$?
  
  \begin{checkboxes}
  \choice $\overline{X} \sim N(5,\frac{5}{30})$ y $\sum_i X_i \sim N(150, 150)$
  \choice $\overline{X} \sim N(5,\frac{25}{30})$ y $\sum_i X_i \sim N(150, 750)$
  \choice $\overline{X} \sim N(5,\frac{5}{30})$ y $\sum_i X_i \sim N(150, 750)$
  \choice $\overline{X} \sim N(5,\frac{25}{30})$ y $\sum_i X_i \sim N(150, 150)$
  \choice No es posible saber con los datos
  \end{checkboxes}
  

  
  \end{questions}
  
  
  
  \end{document}
  