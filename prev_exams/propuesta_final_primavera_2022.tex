%%% Template originaly created by Karol Kozioł (mail@karol-koziol.net) and modified for ShareLaTeX use

\documentclass[addpoints]{exam}

\usepackage[T1]{fontenc}
\usepackage[utf8]{inputenc}
\usepackage{graphicx}
\usepackage{xcolor}

\renewcommand\familydefault{\sfdefault}
\usepackage{tgheros}

\usepackage{amsmath}
\usepackage{amssymb,amsthm,textcomp}
\usepackage{enumerate}
\usepackage{multicol}
\usepackage{tikz}
\usepackage[spanish]{babel}

\usepackage{geometry}
\geometry{left=25mm,right=25mm,bindingoffset=0mm, top=20mm,bottom=20mm}


\linespread{1.3}

\newcommand{\linia}{\rule{\linewidth}{0.5pt}}

% custom theorems if needed
\newtheoremstyle{mytheor}
{1ex}{1ex}{\normalfont}{0pt}{\scshape}{.}{1ex}
{{\thmname{#1 }}{\thmnumber{#2}}{\thmnote{ (#3)}}}
  
  \theoremstyle{mytheor}
  \newtheorem{defi}{Definition}
  
  % my own titles
  \makeatletter
  \renewcommand{\maketitle}{
    \begin{center}
    \vspace{2ex}
    {\huge \textsc{\@title}}
    \vspace{1ex}
    \\
    \linia\\
    \@author \hfill \@date
    \vspace{4ex}
    \end{center}
  }
  \makeatother
  %%%
  
  % custom footers and headers
  %\usepackage{fancyhdr}
  %\pagestyle{fancy}
  \lfoot{Assignment \textnumero{} 5}
  \cfoot{}
  \rfoot{Page \thepage}
  %\renewcommand{\headrulewidth}{0pt}
  %\renewcommand{\footrulewidth}{0pt}
  
  % code listing settings
  \usepackage{listings}
  \lstset{
    language=Python,
    basicstyle=\ttfamily\small,
    aboveskip={1.0\baselineskip},
    belowskip={1.0\baselineskip},
    columns=fixed,
    extendedchars=true,
    breaklines=true,
    tabsize=4,
    prebreak=\raisebox{0ex}[0ex][0ex]{\ensuremath{\hookleftarrow}},
    frame=lines,
    showtabs=false,
    showspaces=false,
    showstringspaces=false,
    keywordstyle=\color[rgb]{0.627,0.126,0.941},
    commentstyle=\color[rgb]{0.133,0.545,0.133},
    stringstyle=\color[rgb]{01,0,0},
    numbers=left,
    numberstyle=\small,
    stepnumber=1,
    numbersep=10pt,
    captionpos=t,
    escapeinside={\%*}{*)}
  }
  
  %%%----------%%%----------%%%----------%%%----------%%%
  
  \begin{document}
  
  \title{Propuesta Final - Estadística II}
  
  \author{ITAM, Primavera 2022}
  
  \date{08/05/2022}
  
  \maketitle
  
  \section*{Propiedades de estimadores}

  \begin{questions}
  
  \question Se sabe que el costo de televisores (X) de última generación en la industria tiene la siguiente distribución:
  
  $$ f(x; \theta) =  \frac {3 \theta^3}{x^4} \quad si \quad  x > \theta $$
  
  con $\theta$ un parámetro desconocido. Una empresa productora de televisores desea estimar el parámetro para establecer su lista de precios, por lo que toma una muestra de 3 televisores (iid) y registra el costo, proponiendo a $\hat{\theta}$ como estimador de $\theta$:
  
  $$ \hat{\theta}  = c \bar{X} $$
  
  \begin{enumerate}
  \item Determine el valor de $c$ tal que $\hat{\theta}$ es insesgado. (c = 2/3)
  \item Usando el valor $c$ del inciso anterior, determine su ECM (ECM = $\frac{1}{9} \theta^2$)
  \end{enumerate}
  
  
  \question La distancia promedio que recorre un ave migratoria en cientos de kilómetros tiene la siguiente distribución:
  
  $$ f(x; \theta) =  \frac {2x}{\theta^2} \quad si \quad  0 < x < \theta, \quad \theta > 0$$
  
  Para estimar el parámetro $\theta$ se toma una muestra de 3 aves (iid) y se propone el estimador: 
  $$ \hat{\theta}  = c \bar{X} $$
  
  \begin{enumerate}
  \item Determine el valor de $c$ tal que $\hat{\theta}$ es insesgado. (c = 3/2)
  \item Usando el valor $c$ del inciso anterior, determine su ECM (ECM = $\frac{1}{24} \theta^2$)
  \end{enumerate}
  
  \end{questions}
  
  \section*{Pruebas de hipótesis paramétricas}
  
  
  \begin{questions} 
  \question  El Director General de una empresa de consumo tiene la creencia que la inversión en marketing (X) tiene una relación positiva con las ventas de su empresa (Y). Para verificar su creencia selecciona una muestra de 12 semanas de datos de inversión en marketing y de ventas (los datos se presentan en millones). Los resultados son los siguientes:
  
  $$ \sum_{i=1}^{12} x_i = 111 \quad \sum_{i=1}^{12} x_i^2 = 1,277 \quad \sum_{i=1}^{12} y_i = 462 \quad \sum_{i=1}^{12} y_i^2 = 26,944 \quad \sum_{i=1}^{12} x_i y_i = 5,585 $$
  
Utilicé una prueba de hipótesis paramétrica para probar si existe evidencia de una asociación positiva entre la inversión en marketing y las ventas (use $\alpha = 0.05$).

\begin{enumerate}
  \item Plantear $H_1$ y $H_0$ 
  \item Determina la Región de Rechazo y el estadístico a utilizar
  \item Concluya utilizando los datos de la muestra proporcionada ( r = 0.8664, $t_{10}$ = 5.4867)
  \item Determinal el valor-p de la prueba 
\end{enumerate}

  \question El secretario de economía de un país de latinoamérica desea saber si existe una relación lineal entre la tasa de desempleo (X) y la tasa de inflación de su país (Y). Para esto toma información de 10 meses:
  
  $$ \sum_{i=1}^{10} x_i = 0.292 \quad \sum_{i=1}^{10} x_i^2 = 0.012294  \quad \sum_{i=1}^{10} y_i = 0.83 \quad \sum_{i=1}^{10} y_i^2 = 0.0883 \quad \sum_{i=1}^{10} x_i y_i = 0.03185 $$
  
  
Utilicé una prueba de hipótesis paramétrica para probar si existe evidencia de una asociación positiva entre la tasa de desempleo y la tasa de inflación (use $\alpha = 0.025$) 

  \begin{enumerate}
  \item Plantear $H_1$ y $H_0$ 
  \item Determina la Región de Rechazo y el estadístico a utilizar
  \item Concluya utilizando los datos de la muestra proporcionada ( r = 0.8904, $t_8$ = 5.5328)
  \item Determinal el valor-p de la prueba 
\end{enumerate}
  


  
 
  
  \end{questions}

  \end{document}
  