%%% Template originaly created by Karol Kozioł (mail@karol-koziol.net) and modified for ShareLaTeX use

\documentclass[addpoints]{exam}

\usepackage[T1]{fontenc}
\usepackage[utf8]{inputenc}
\usepackage{graphicx}
\usepackage{xcolor}

\renewcommand\familydefault{\sfdefault}
\usepackage{tgheros}

\usepackage{amsmath}
\usepackage{amssymb,amsthm,textcomp}
\usepackage{enumerate}
\usepackage{multicol}
\usepackage{tikz}
\usepackage[spanish]{babel}

\usepackage{geometry}
\geometry{left=25mm,right=25mm,bindingoffset=0mm, top=20mm,bottom=20mm}


\linespread{1.3}

\newcommand{\linia}{\rule{\linewidth}{0.5pt}}

% custom theorems if needed
\newtheoremstyle{mytheor}
{1ex}{1ex}{\normalfont}{0pt}{\scshape}{.}{1ex}
{{\thmname{#1 }}{\thmnumber{#2}}{\thmnote{ (#3)}}}
  
  \theoremstyle{mytheor}
  \newtheorem{defi}{Definition}
  
  % my own titles
  \makeatletter
  \renewcommand{\maketitle}{
    \begin{center}
    \vspace{2ex}
    {\huge \textsc{\@title}}
    \vspace{1ex}
    \\
    \linia\\
    \@author \hfill \@date
    \vspace{4ex}
    \end{center}
  }
  \makeatother
  %%%
  
  % custom footers and headers
  %\usepackage{fancyhdr}
  %\pagestyle{fancy}
  \lfoot{Examen parcial \textnumero{} 1}
  \cfoot{}
  \rfoot{Page \thepage}
  %\renewcommand{\headrulewidth}{0pt}
  %\renewcommand{\footrulewidth}{0pt}
  
  % code listing settings
  \usepackage{listings}
  \lstset{
    language=Python,
    basicstyle=\ttfamily\small,
    aboveskip={1.0\baselineskip},
    belowskip={1.0\baselineskip},
    columns=fixed,
    extendedchars=true,
    breaklines=true,
    tabsize=4,
    prebreak=\raisebox{0ex}[0ex][0ex]{\ensuremath{\hookleftarrow}},
    frame=lines,
    showtabs=false,
    showspaces=false,
    showstringspaces=false,
    keywordstyle=\color[rgb]{0.627,0.126,0.941},
    commentstyle=\color[rgb]{0.133,0.545,0.133},
    stringstyle=\color[rgb]{01,0,0},
    numbers=left,
    numberstyle=\small,
    stepnumber=1,
    numbersep=10pt,
    captionpos=t,
    escapeinside={\%*}{*)}
  }
  
  %%%----------%%%----------%%%----------%%%----------%%%
  
  \begin{document}
  
  \title{Parcial 1 - Estadística II}
  
  \author{ITAM, Primavera 2020}
  
  \date{10/02/2021}
  
  \maketitle
  
  \section*{Instrucciones}
  
El examen es para resolver en casa. Se debe contestar individualmente y entregarse a más tardar a las 23:59 del jueves 11 de febrero. La entrega será via email a la dirección: salvador.garcia.gonzalez@itam.mx (la misma que aparece en comunidad). El examen cuenta dos secciones, la segunda con   preguntas a desarrollar. Se debe cuidar la formalidad al escribir los resultados, ya que es parte de la calificación del problema. En caso de no tener el desarrollo de la pregunta, o bien se llegué a la respuesta sin una justificación se podrá anular la respuesta. Cualquier práctica fraudulenta será sancionada de acuerdo al reglamento del departamento. 
  
  \section*{Seccion A: Preguntas de teoría (45 pts)}
  
  \begin{questions}
  
  \question \textbf{Propiedades varianza y esperanza}
  
Sea X una variable aleatoria con E(X) = $\mu$ y V(X) = $\sigma^2$. Y se distribuye igual a X. Se toma n muestras de X y m de Y. Calcule las siguientes expresiones:
  
  \begin{enumerate}
  \item Usando X calcule la esperanza de la media muestral: $E(\bar{X})$
  \item Usando X calcule la esperanza de la varianza muestral (sesgada): $E(S_n^2) $
  \item Usando X calcule la esperanza de la varianza muestral (insesgada): $E(S^2)$
  \item Usando X calcule la varianza de la media muestral: $V(\bar{X})$
  \item Usando X y Y calcule la esperanza de la diferencia de medias muestrales: $E(\bar{X} - \bar{Y})$
  \item Usando X y Y calcule la varianza de la suma de medias muestrales: $V(\bar{X} + \bar{Y})$
  \end{enumerate}
  
  \question \textbf{Teoría}
  Explique los siguientes puntos:
  \begin{enumerate}
  \item Describe la diferencia entre un estimador y un estadístico y como se relacionan
  \item Describe la diferencia entre un parámetro y un estimador y como se relacionan
  \item Explique el proceso de inferencia estadística y de un ejemplo de inferencia
  \item Justifique la razón por la cual nos interesa conocer la distribución de los estadísticos
  \item Explique las diferencias entre distribución muestral exacta y aproximada
  \end{enumerate}
  
  \question \textbf{Verdadero/Falso}
  Justifique si las siguientes afirmaciones son verdaderas o falsas y justifique 
  \begin{enumerate}
  \item Si se tiene una población de tamaño 10 y se toman muestras de tamaño 1, la distribución de muestreo con reemplazo y sin reemplazo son iguales. 
  \item El teorema central del limite se puede utilizar siempre y cuando tengamos muestras de tamaño pequeñas.
  \item Suponga una muestra de tamaño 40. Para encontrar la distribución de $\bar{X}$, donde cada $X_i$ se distribuye normal, es necesario utilizar el TCL.
  \item La importancia del Teorema Central del Límite radica en encontrar la distribución de estadísticos solamente cuando las distribuciones son normales, exponenciales y poisson. 
  \end{enumerate}
  
  \end{questions}
  
  \section*{Seccion B: Preguntas a desarrollar (55 pts)}
  
  
  \begin{questions} 
  \question (20 pts) Suponga que una variable aleatoria puede tomar los valores \{1,3,5\}. Considere muestras de tamaño 2 con reemplazo.
  
  \begin{enumerate}
  \item Calcule $E(X)$ y $V(X)$
  \item Obtenga distribución de muestreo de $\overline{X}$, $E(\overline{X})$ y $V(\overline{X})$
  \item Obtenga distribución de muestreo de $S^2$, $E(\overline{S^2})$ y $V(\overline{S^2})$
  \item Calcule la probabilidad que $S^2$ tome valores entre 1 y 3
  \item ¿Las distribuciones de $S^2$ y $\overline{X}$ son distribuciones exactas?
  Justifique
  \end{enumerate}
  
  \question (15 pts) Suponga que se tienen 5 vehiculos $V_1$, $V_2$, $V_3$, $V_4$, $V_5$. Los vehículos $V_3$ y $V_4$ requieren reparación.

  \begin{enumerate}
  \item Obtenga la distribución de muestreo de la proporción de vehículos $\hat{p}$ que necesitan reparación, si se toman muestras de dos vehiculos sin reemplazo
  \item Calcule el valor esperado de la distribución de muestreo y la varianza de la distribución de muestreo $E(\hat{p})$ y $V(\hat{p})$
  \end{enumerate}
  
  
  \question (15 pts) 
  En la torre de control de un aeropuerto se contaron cuantos aviones despegaban durante 40 periodos de una hora seleccionados al azar durante un mes. Supongase que la distribución del número de aviones (X) que despegan por hora es normal con $\mu = 50$ y desviación estándar $\sigma=7$.
  
  \begin{enumerate}
  \item ¿Cuál es la probabilidad de que la media muestral para n = 40 periodos de una hora sea mayor a 55?
  \item Suponga que n = 5, ¿Cuál es la probabilidad que $\bar{X}$ sea mayor que 55?
  \item ¿La distribución presentada para n = 40 es una distribución exacta o aproximada? ¿Para n = 5 es una distribución exacta o aproximada?
  \item ¿Cuál es la probabilidad que el número total de aviones para un periodo de 4 horas sea mayor que 180?
  \end{enumerate}

  \question (15 pts)
  El peso de unas computadoras que se distribuye de forma normal con media 10 kgs y desviación estándar de 3. 
  \begin{enumerate}
  \item ¿Cuál es la probabilidad que un producto seleccionado al azar pese más de 12 kg?
  \item ¿Si se toma una muestra de 9 productos, ¿Cuál es la probabilidad que la media de la muestra sea menor a 10?
  \end{enumerate}

  \end{questions}

  \end{document}
  